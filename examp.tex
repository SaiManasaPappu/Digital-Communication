\documentclass{beamer}
\mode<presentation>
\usepackage{amsmath}
\usepackage{amssymb}
%\usepackage{advdate}
\usepackage{adjustbox}
\usepackage{subcaption}
\usepackage{enumitem}
\usepackage{multicol}
\usepackage{listings}
\usepackage{gensymb}
\usepackage{url}
\usepackage{gensymb}

\def\UrlBreaks{\do\/\do-}
\usetheme{Boadilla}
\usecolortheme{lily}
\setbeamertemplate{footline}
{
  \leavevmode%
  \hbox{%
  \begin{beamercolorbox}[wd=\paperwidth,ht=2.25ex,dp=1ex,mathright]{author in head/foot}%
    \insertframenumber{} / \inserttotalframenumber\hspace*{2ex} 
  \end{beamercolorbox}}%
  \vskip0pt%
}
\setbeamertemplate{navigation symbols}{}

\providecommand{\nCr}[2]{\,^{#1}C_{#2}} % nCr
\providecommand{\nPr}[2]{\,^{#1}P_{#2}} % nPr
\providecommand{\mbf}{\mathbf}
\providecommand{\pr}[1]{\ensuremath{\Pr\mathmathleft(#1\mathmathright)}}
\providecommand{\qfunc}[1]{\ensuremath{Q\mathleft(#1\mathright)}}
\providecommand{\sbrak}[1]{\ensuremath{{}\mathleft[#1\mathright]}}
\providecommand{\lsbrak}[1]{\ensuremath{{}\mathleft[#1\mathright.}}
\providecommand{\rsbrak}[1]{\ensuremath{{}\mathleft.#1\mathright]}}
\providecommand{\brak}[1]{\ensuremath{\mathleft(#1\mathright)}}
\providecommand{\lbrak}[1]{\ensuremath{\mathleft(#1\mathright.}}
\providecommand{\rbrak}[1]{\ensuremath{\mathleft.#1\mathright)}}
\providecommand{\cbrak}[1]{\ensuremath{\mathleft\{#1\mathright\}}}
\providecommand{\lcbrak}[1]{\ensuremath{\mathleft\{#1\mathright.}}
\providecommand{\rcbrak}[1]{\ensuremath{\mathleft.#1\mathright\}}}
\theoremstyle{remark}
\newtheorem{rem}{Remark}
\newcommand{\sgn}{\mathop{\mathrm{sgn}}}
\providecommand{\abs}[1]{\mathleft\vert#1\mathright\vert}
\providecommand{\res}[1]{\Res\displaylimits_{#1}} 
\providecommand{\norm}[1]{\lVert#1\rVert}
\providecommand{\mtx}[1]{\mathbf{#1}}
\providecommand{\mean}[1]{E\mathleft[ #1 \mathright]}
\providecommand{\fourier}{\overset{\mathcal{F}}{ \mathrightmathleftharpoons}}
%\providecommand{\hilbert}{\overset{\mathcal{H}}{ \mathrightmathleftharpoons}}
\providecommand{\system}{\overset{\mathcal{H}}{ \longmathleftmathrightarrow}}
	%\newcommand{\solution}[2]{\textbf{Solution:}{#1}}
%\newcommand{\solution}{\noindent \textbf{Solution: }}
\providecommand{\dec}[2]{\ensuremath{\overset{#1}{\underset{#2}{\gtrless}}}}
\newcommand{\myvec}[1]{\ensuremath{\begin{pmatrix}#1\end{pmatrix}}}
\let\vec\mathbf


\lstset{
%language=C,
frame=single, 
breaklines=true,
columns=fullflexible
}

\numberwithin{equation}{section}

\title{\LARGE  {\bf POWER  ELECTRONICS}}
\author{{Sai Manasa Pappu   EE17BTECH11036}\\  { Sahir Bansal  EE17BTECH11035}}

\date{\today}

\begin{document}

\begin{frame}
\titlepage
\end{frame}


\begin{frame}
\frametitle{Problem Statement}
Python script to find the output of the 4th order low pass filter, with input as obtained from result of 3.1. 
\bigbreak
\end{frame}

%\subsection{Literature}
\section{Solution}
\subsection{Centre}
\begin{frame}
\frametitle{Filter Response}
%\framesubtitle{Literature}
\bigbreak
\bigbreak
\bigbreak
Fourth order low pass filter cutoff frequency = 50 Hz. Note that this frequency is same as the input frequency. 
\bigbreak
Hence input will be attenuated by 3dB. 
\begin{align}
%\vec{x}^T\vec{x}-2\vec{O}^T\vec{x} +F = 0
V_{out} (peak) = \frac{1}{\sqrt{2}}V_{in} (peak) 
\end{align}
%
Phase at f = 50 Hz is 4 x -45{\degree} = -180\degree

\begin{align}
\angle H(f) = -180 \degree 
\end{align}
We can also verify the same by plotting magnitude and phase response of the filter in python.
\end{frame}


\begin{frame}
\begin{figure}
\frametitle{Magnitude Response}
\includegraphics[width=1\columnwidth ]{magnitude.png}
\label{magnitude.png}
\end{figure}
\end{frame}

\begin{frame}
\frametitle{Phase Response}
\begin{figure}
\centering
\includegraphics[width=1\columnwidth ]{phase.png}
\end{figure}
\end{frame}

\begin{frame}{}
\frametitle{Output}
    Hence output frequency will be same as input frequency, peak voltage reduces by a factor of ${\sqrt{2}}$ and is out of phase with input.
    \bigbreak
    
\begin{align}
V_{out} = \frac{1}{\sqrt{2}} V_{in} \angle 180 \degree
\end{align}
    
\begin{align}
V_{out} = -0.707V_{m}sin(wt)
\end{align}
    
\end{frame}




\begin{frame}
\frametitle{Output}
\begin{figure}
\centering
\includegraphics[width=1\columnwidth ]{out.png}
\end{figure}
\end{frame}


\begin{frame}{Github Link for Python Code }
\url{https://github.com/SaiManasaPappu/Digital-Communication/blob/master/3.5.py}
    
\end{frame}


\end{document}
